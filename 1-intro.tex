Квантовая сцепленность является следствием множества интересных физических явлений.
Квантовая криптография\cite{PhysRevLett.67.661}, парадокс Эйнштейна — Подольского — Розена\cite{PhysRevLett.69.2881}, квантовая телепортация\cite{PhysRevLett.70.1895} и многие другие эффекты основываются на квантовой сцепленности.
Но, к сожалению, обычно очень трудно создавать, поддерживать и манипулировать сцепленными состояниями в лабораторных условиях.
На самом деле любая система обычно подвержена воздействию внешних шумов и взаимодействию с окружающей средой.
Эти эффекты превращают сцепленность в чистом состоянии в смешанное состояние или шумовую сцепленность.
Проблема отделимости, то есть характеристика смешанных сцепленных состояний, весьма нетривиальна и до сих пор не решена.
Даже, казалось бы, простой вопрос: "является ли данное состояние сцепленным и содержит ли оно квантовые корреляции, или оно сепарабельно и не содержит никаких квантовых корреляций?" не имеет точного ответа на данный момент.

Данная работа поможет немного снять завесу неизвестного, поможет  ответить на вопрос является ли представленное состояние истинно сцепленным или нет.
В работе представлен общий метод описания истинной многочастичной сцепленности.
Он основан на так называемых свидетелях сцепленности\cite{dariusz_2014}(на англ. entanglement witness), операторах, обнаруживающие наличие сцепленности.
Для нахождения оптимального оператора сцепленности ставиться задача полуопределенного программирования(на англ. semidefinite programming).
Решая её мы сможем с некоторой точностью сказать является ли представленное состояние сцепленным или нет.
В результате получим критерий, который можно рассматривать как обобщение критерия Переса-Городецкого\cite{criterion-peres-horodecki} на многочастичный случай.
Теоретические рассуждения будут подкреплены практикой.
Полученный метод определения истинной сцепленности будет применен к многочастичным цепочкам состояний Вернера и изотропным состояниям разной формы.
Задача полуопределенного программирования для поиска свидетеля сцепленности будет решена численно.
Опираясь на аналитические рассчеты других статей на эту тему, будет проведено сравнение полученных результатов с теорией.
