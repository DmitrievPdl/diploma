\section{Конечная цепочка состояний Вернера}
Рассмотрим задачу поиска границ истинной сцепленности смешанных состояний аналитическим методом.
Похожая техника определения теоретических границ была в статье \cite{Contreras_Tejada_2022}, только там рассматривали изотропные состояния, но аналогичные рассуждения, как мы увидим в дальнейшем, можно применить и для состояния Вернера.

Рассмотрим \textbf{состояния Вернера} \cite{PhysRevA.40.4277} - семейство состояний на пространстве $\mathbb{C}^d\otimes\mathbb{C}^d$, которые инвариантные относительно унитарных преобразований $U \otimes U$, $\rho_{wer} = (U \otimes U)\rho_{wer} (U^{\dag} \otimes U^{\dag})$. Семейство можно параметризовать следующим образом
\begin{equation}\label{def-werner-state}
    \rho_{wer}(p,d) = \frac1{d^2 + pd}\left(\mathbb{1} + p\sum_{i,\,j=0}^{d-1}\,\ket{i,\,j}\bra{j,\,i}\right),
\end{equation}
где $d$ - размерность и $p \in [-1, 1]$ - вещественны параметр состояния.
Теоретические границы сепарабельности и сцепленности данного состояния известны
\begin{table}[ht]
\centering
\begin{tabular}{ l | c }
  сепарабельно & $ -1 \leq p \leq -\frac{1}{d}$ \\
  \hline
  сцеплено & $ -\frac{1}{d} \leq p \leq -1$ \\
\end{tabular}
\caption{Границы сцепленности и сепарабельности для состояния Вернера}
\label{table:theor-bound-werner}
\end{table}
Дальнейшая цель найти границы для системы произвольного количества состояний Вернера, которые образуют цепочку. 

Заметим, что \ref{def-werner-state} можно представить в следующем виде
\begin{equation}
\begin{split}
& \rho_{wer}(p, d) = 
    \frac{1}{d^2 + pd}
    \left(
    (1 + p) \mathbb{1} - 2p \Pi_{\mathcal A} 
    \right) = \\
& = \frac{1}{d^2 - pt}
    \left(
    (1 - t) \mathbb{1} + 2t \Pi_{\mathcal A} 
    \right)  \\
\end{split}
\end{equation}
где $t = -p$ замена переменной и $\Pi_{\mathcal A} = \frac{\mathbb{1}-\mathrm{SWAP}}2$ проектор на антисимметричное подпространство пространства $\mathbb{C}^d\otimes\mathbb{C}^d$ и $\mathrm{SWAP} = \sum_{i,\,j=0}^{d-1}\,\ket{i,\,j}\bra{j,\,i}$ оператор, который меняет местами кубиты.

\begin{figure}[h]
\center{
    \begin{overpic}[width=0.6\textwidth]{img/werner-chain.png}
    \end{overpic}
}
\caption{Цепочка из $N$ состояний Вернера.}
\label{img:werner-chain}
\end{figure}
Рассмотрим цепочку из $N$ состояний Вернера.
Матрица плотности данного состояния будет иметь следующий вид
\begin{equation}
\begin{split}
& \rho_{chain} = \rho_1 \otimes ... \otimes \rho_N = \\
& = \frac{1}{d^2 - td}
((1 - t) \mathbb{1} + 2t\Pi_{\mathcal A})
\otimes ... \otimes 
\frac{1}{d^2 - td}
((1 - t) \mathbb{1} + 2t\Pi_{\mathcal A}).
\end{split}
\end{equation}
Раскроем произведения, сгруппировав слагаемые
\begin{equation}
\begin{split}
& \rho_{chain} = 
\left(\frac{2t}{d^2 - td}\right)^N \Pi_{\mathcal A} \otimes ... \otimes \Pi_{\mathcal A}
+
\sum\limits_{i_1 = 1}^{N} \frac{(1-t)(2t)^{N-1}}{(d^2 - 2t)^N} \mathbb{1}^{i_1} \otimes \prod_{j \neq i_1} \otimes \Pi_{\mathcal A}^{j} + \\
&
+\sum\limits_{i_1, i_2 = 1, i_1 \neq i_2}^{N} \frac{(1-t)^2(2t)^{N-2}}{(d^2 - 2t)^N} \mathbb{1}^{i_1} \otimes \mathbb{1}^{i_2} \otimes \prod_{j \neq i_1,i_2} \otimes \Pi_{\mathcal A}^{j}  + ...,
\end{split}
\end{equation}
где $i_k$ и $j$ нумеруют порядок произведении кронекера. Далее заметим, что все слагаемые с двумя или более единичными операторами $\mathbb{1}$ являются бисепарабельными.
Действительно, можно выбрать бипартицию, которая будет делить состояние не две части.
Каждая из этих частей будет начинаться с единичного оператора.
Относительно такой бипартици будет выполнено условие бисепарабельности \ref{bisep-state-def}.
Причем в данном случае формула будет содержать одно слагаемое $p=1$.
Поэтому их можно отбросить, заменив их на $\rho^{bs}$.

В итоге получим следующее выражение для матрицы плотности цепочки
\begin{equation}
\begin{split}
& \rho_{chain} = 
\left(\frac{2t}{d^2 - td}\right)^N \Pi_{\mathcal A} \otimes ... \otimes \Pi_{\mathcal A}
+
\sum\limits_{i_1 = 1}^{N} \frac{(1-t)(2t)^{N-1}}{(d^2 - 2t)^N} \mathbb{1}^{i_1} \otimes \prod_{j \neq i_1} \otimes \Pi_{\mathcal A}^{j} + \rho^{bs}.
\end{split}
\end{equation}
Задача свелась к исследованию сцепленности для первых двух слагаемых.
В дальнейшем будем отбрасывать дополнительны индекс $i_1 = i$, так как $i$ будет встречаться только один раз.
Покажем, что их сумму можно свести к сумме состояний Вернера.
Действительно, внесем первое слагаемое под сумму и вынесем общие операторы проекции $\Pi_{\mathcal A}$ за скобку 
\begin{equation}
\begin{split}
& \rho_{chain} = 
\sum\limits_{i=1}^{i=N}
\left[
\frac{1}{N}\left(\frac{2t}{d^2 - td}\right)^N
\Pi_{\mathcal A}^{i} + 
\frac{(1-t)(2t)^{N-1}}{(d^2 - 2t)^N} \mathbb{1}^{i}
\right] 
\otimes \prod_{j \neq i} \otimes \Pi_{\mathcal A}^{j} + \rho^{bs}.
\end{split}
\end{equation}
В центре квадратных скобок получили, с точностью до коэффициентов, состояние Вернера. Справа идет произведение кронекера с операторами проекции $\Pi_{\mathcal A}$ на антисимметричное пространство, которое является сцепленным. Обозначим его как $\rho^{ent}$. Тогда мы придем к более простому виду матрицы плотности
\begin{equation}\label{rho-chain-end}
\begin{split}
& \rho_{chain} = 
\sum\limits_{i=1}^{i=N}
\left[
\frac{1}{N}\left(\frac{2t}{d^2 - td}\right)^N
\Pi_{\mathcal A}^{i} + 
\frac{(1-t)(2t)^{N-1}}{(d^2 - 2t)^N} \mathbb{1}^{i}
\right] \otimes
\rho^{ent} + \rho^{bs}.
\end{split}
\end{equation}

В итоге задача свелась к определению сцепленности выражения в квадратных скобках \ref{rho-chain-end}.
Если оно окажется сепарабельным, то и $\rho_{chain}$ окажется бисепарабельным. В противном случае, $\rho_{chain}$ окажется истинно сцепленным.
Как уже упоминалось, выражение в квадратных скобках \ref{rho-chain-end} представляет собой, с точностью до коэффициентов, состояние Вернера. Вынесем общий коэффициент $\frac{(2t)^{N-1}}{(d^2 - td)^N}$, который является положительным, согласно условию на допустимые параметры $p\in[-1,1]$ и определению $t = -p$, и не влияет на сцепленность
\begin{equation}
\begin{split}
& \frac{2t}{N} \Pi_{\mathcal A} + (1-t)\mathbb{1} = \frac{2t}{N} \left( \frac{\mathbb{1} - \mathrm{SWAP}}{2} \right) + (1-t)\mathbb{1} = \\
& = \left( 1-t + \frac{t}{N}\right) \mathbb{1} - \frac{t}{N} \mathrm{SWAP} \sim \\
& \sim \mathbb{1} - \frac{t}{N( 1-t + \frac{t}{N})} \mathrm{SWAP}.
\end{split}
\end{equation}
В итоге мы получили, с точностью до коэффициента нормировки, состояние Вернера.
Задача свелась к исследованию границ бисепарабельности для данного состояния.
Используя таблицу \ref{table:theor-bound-werner} с теоретическими границами  для состояния Вернера, определяем граничное значение для цепочки
\begin{equation}
    \frac{t}{N( 1-t + \frac{t}{N})} = \frac{1}{d}
\end{equation}
Вернёмся снова к параметру $p = -t$ и выразим границу через него
\begin{equation}\label{sep-bound}
    p_{bound} = - \frac{N}{N + d - 1}.
\end{equation}
Для всех $p < p_{bound}$ цепочка из $N$ состояний Вернера является сцепленной.