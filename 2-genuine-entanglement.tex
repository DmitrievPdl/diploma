
\section{Определение}
На примере трехчастичного состояния рассмотрим основные термины, которые будут использоваться в  данной работе.  В случае трех частиц $A$, $B$ и $C$ имеет место три бипартиции, которые мы обозначим как $A | BC$, $B | AC$ и $C | AB$.
Пусть трехчастичное состояние задается матрицей плотность $\rho$. Это состояние называют \textbf{сепарабельным} по отношению к бипартиции $A | BC$, если его можно представить в виде

\begin{equation}\label{sep-def}
\rho = \sum\limits_k q_k \ket{\phi_A^k} \bra{\phi_A^k}
\otimes \ket{\psi_{BC}^k} \bra{\psi_{BC}^k},
\end{equation}
где $q_k$ - положительные веса $\sum\limits_{k}q_k = 1$, а $\ket{\phi_A^k} \bra{\phi_A^k}$ и $\ket{\psi_{BC}^k} \bra{\psi_{BC}^k}$ представляют собой матрицы плотности для каждой из подсистем в заданной бипартиции $A | BC$. Будем обозначать сепарабельное состояние таким образом $\rho^{sep}_{A | BC}$. Аналогично можно выписать остальные выражениея для сепарабельных состояний относительно оставшихся бипартиций $\rho_{B | AC}^{sep}$ и $\rho_{C | AB}^{sep}$.

Состояние называется \textbf{бисепарабельным}, если можно представить в виде смеси сепарабельных состояний, каждое из которых сепарабельно относительно различных бипартиций

\begin{equation}\label{bisep-state-def}
    \rho^{bs} = p_1 \rho_{A | BC}^{sep} +
    p_2 \rho_{B | AC}^{sep} +
    p_3 \rho_{C | AB}^{sep}.
\end{equation}
В противном случае, если состояние не является бисепарабельным, то его называют \textbf{истинно сцепленным}.

Для того чтобы доказать, что состояние является истинно сцепленным, достаточно доказать, что оно не является бисепарабельным. Для систем малой размерности существует критерий Переса-Городетского \cite{criterion-peres-horodecki}.
Согласно ему, для гильбертовых пространств $H=\mathbb{C}^2\otimes\mathbb{C}^2$ и 
$H=\mathbb{C}^2\otimes\mathbb{C}^3$ заданное состояние $\rho$ является сепарабельным тогда и только тогда, когда её 
частичное транспонирование(транспонирование относительно одной из партиций, 
подробнее можно найти в конце работы\ref{appendix:partial_transpose}) 
является положительно определенным или, другими словами, 
не имеет отрицательных собственных значений. Далее в работе это свойство будет обозначаться следующим образом $A \geq 0$, 
где $A$ положительно определенная матрица.
Для больших размерностей критерий Переса-Городетского не работает, поэтому используют другой механизм детектирования истинной запутанности, который связан с оператором \textbf{свидетелем сцепленности}.


\section{Свидетель сцепленности}
\begin{definition}\label{ew-def}
Оператор $W = W^{\dag}$, который действует на гильбертовом пространстве $H = H_1 \otimes H_2 \otimes ... \otimes H_N$, называется \textbf{свидетелем сцепленности}, если он удовлетворяет следующим свойствам:
\begin{equation}
    \begin{split}
        & \text{(I) } \forall \rho^{sep}: \textbf{Tr}(W\rho^{sep}) \geq 0 \\
        & \text{(II) } W \geq 0 \\
        & \text{(III) }\textbf{Tr}(W) = 1 
    \end{split}
\end{equation}

\end{definition}

Первое условие (I) говорит нам о том, что с помощью среднего значения свидетеля сцепленности можно определить сцепленность состояния. Его еще можно представить в виде $\rho^{sep}: \textbf{Tr}(W\rho^{sep}) = \langle W \rangle_{\rho} \geq 0$. Второе (II) означает, что каждый свидетель сцепленности что-то обнаруживает, поскольку, он обнаруживает проектор на подпространство, соответствующий отрицательным собственным значениям $W$. Третье свойство (III) — это просто условие нормировки. 


Рассмотрим свидетель сцепленности $W$, для определения основных свойств и особенностей данного оператора нам понадобятся несколько определений:

\begin{itemize}
  \item Обозначим $D_W = \{ \rho \geq 0: \langle W \rangle_{\rho} < 0 \}$ множество состояний, которые сможет детектировать $W$.

  \item Пусть определены два свидетеля сцепленности $W_1$ и $W_2$, тогда $W_2$ считается \textbf{оптимальнее(finer)} чем $W_1$, если $D_{W_1} \subseteq D_{W_2}$, или, другими словами, состояния, которые может детектировать $W_2$, также может детектировать $W_1$.

  \item \textbf{Оптимальным свидетелем сцепленности} называется свидетель сцепленности, для которого не существует свидетеля тоньше.

  \item Обозначим $P_W = \{ \phi \in H: \bra{\phi} W \ket{\phi} = 0 \}$ множество состояний, для которых среднее значение свидетеля сцепленности $W$ обращается в $0$.
\end{itemize}


Обратим внимание, какую роль играют состояний, находящиеся в $P_W$, в относительной сцепленности. Пусть есть некоторый свидетель сцепленности $W$, который детектирует состояние $\rho$, тогда он также будет детектировать состояние $\rho + \rho^{p}$, где
\begin{equation}
    \rho^{p} = \sum\limits_{k} q_k \ket{\phi}
    \bra{\phi}, \phi \in P_W.
\end{equation}
Это означает, что если мы добавим сколь угодно малое количество $\rho$ к $\rho^{p}$, то полученное состояние уже станет сцепленным. Таким образом, структура множеств $P_W$ характеризует границу между сепарабельными и сцепленными состояниями. Фактически в дальнейшем нам станет ясно, что можно ограничится структурой множества $P_W$, не рассматривая множество $D_W$, соответствующих оптимальных операторов сцепленности.


В дальнейшем можно ограничиться рассмотрением только оптимальных свидетель сцепленности. Для этого нам необходим критерий, согласно которому мы сможем определить является ли данный свидетель оптимальным. Для этого в этом разделе будут доказаны необходимое и достаточное условие того, что свидетель является оптимальным. Но перед этим необходимо доказать несколько лемм, которые помогут нам в этом. 

\begin{lemma}
Пусть $W_2$ оптимальнее чем $W_1$ и определим
\begin{equation}\label{lambda-lemma}
    \lambda = \inf_{\rho_1 \in D_{W_1}} \abs{\frac{\langle W_1 \rangle_{\rho_1}}{\langle W_2 \rangle_{\rho_1}}}.
\end{equation}
Тогда
\begin{enumerate}
    \item Если $\langle W_1 \rangle_{\rho} = 0$, тогда $\langle W_2 \rangle_{\rho} \leq 0$

    \item Если $\langle W_1 \rangle_{\rho} < 0$, тогда $\langle W_2 \rangle_{\rho} \leq \langle W_1 \rangle_{\rho}$

    \item Если $\langle W_1 \rangle_{\rho} > 0$, тогда $\lambda \langle W_1 \rangle_{\rho} \geq \langle W_2 \rangle_{\rho}$

    \item $\lambda \geq 1$, причем $\lambda = 1$ тогда и только тогда, когда $W_1 = W_2$
\end{enumerate}
\end{lemma}

\underline{Доказательство.} Так как $W_2$ оптимальнее чем $W_1$ мы будем использовать тот факт, что для всех $\rho \geq 0$ если выполняется $\langle W_1 \rangle_{\rho} < 0$, то тогда справедливо неравенство $\langle W_2 \rangle_{\rho} < 0$.

(1) Докажем первое утверждение леммы. Предположим обратное $\langle W_2 \rangle_{\rho} > 0$. Возьмем произвольную матрицу плотности из $D_{W_1}$: $\rho_1 \in D_{W_1}$. Тогда $\forall x \geq 0$, $0 \leq \tilde \rho(x) = \rho_1 + x \rho \in D_{W_1}$. Но при достаточно большом $x$ среднее значение оператора станет положительным $\langle W_2 \rangle_{\tilde \rho(x)} > 0$, чего не может быть так как $\rho \notin D_{W_2}$


(2) Докажем второе утверждение леммы. Определим $\tilde \rho = \rho  + |\langle W_1 \rangle_{\rho}| \mathbb{1} \geq 0$. Заметим, что в этом случае $\langle W_1 \rangle_{\tilde\rho} = 0$. Используя первое утверждение леммы получаем неравенство $\langle W_2 \rangle_{\rho} + \abs{\langle W_1 \rangle_{\rho}} \leq 0$


(3) Докажем третье утверждение. Возьмем $\rho_1 \in D_{W_1}$ и определим $\tilde \rho = \langle W_1 \rangle_{\rho} \rho_1 + \abs{\langle W_1 \rangle_{\rho_1}}\rho \geq 0$, в этом случае $\langle W_1 \rangle_{\tilde \rho} = 0$. Используя первое утверждение леммы получаем $\abs{\langle W_1 \rangle_{\rho_1}}\langle W_2 \rangle_{\rho} \leq \abs{\langle W_2 \rangle_{\rho_1}} \langle W_1 \rangle_{\rho}$. Разделив обе части неравенства на $\abs{\langle W_1 \rangle_{\rho_1}} >0$ и на $\langle W_1 \rangle_{\rho} > 0$ мы получим

\begin{equation}
    \frac{
    \langle W_2 \rangle_{\rho}
    }{
    \langle W_1 \rangle_{\rho}
    }
    \leq
    \abs{
    \frac{
    \langle W_2 \rangle_{\rho_1}
    }{
    \langle W_1 \rangle_{\rho_1}
    }
    }
\end{equation}
Беря инфиниум по отношению к $\rho_1 \in D_{W_1}$ данное неравенство даст необходимый результат.

(4) Докажем последние утверждение леммы. Неравенство $\lambda \geq 1$ следует непосредственно из второго утверждения леммы. Докажем равенство. Если $\lambda = 1$ то используя  (1) и (3) утверждение получим $\forall \rho_v = \ket{\phi}\bra{\phi}: \langle W_1 \rangle_{\rho_v} \geq \langle W_2 \rangle_{W_2}$. Поскольку $\Tr(W_1) = \Tr(W_2)$ тогда $\Tr((W_1 - W_2)\rho_v) = 0$. Теперь для любой матрицы плотности $\rho \geq 0$ можно определить $\tilde \rho(x) = \rho + x \mathbb{1}$ такую, что при достаточно большом $x$ состояние $\tilde \rho(x)$ станет сепарабельным. В этом случае $\langle W_1 \rangle_{\tilde \rho(x)} = \langle W_2 \rangle_{\tilde \rho(x)}$, откуда следует, что $\langle W_1 \rangle_{\rho} = \langle W_2 \rangle_{\rho}$ или $W_1 = W_2$. $\square$


\textbf{Следствие 1} $D_{W_1} = D_{W_2}$ тогда и только тогда, когда $W_1 = W_2$.

\underline{Доказательство} Действительно, определим аналогичное выражение для $\lambda$ из формулы \ref{lambda-lemma}. С другой стороны определим $\tilde \lambda$ как
\begin{equation}
    \tilde \lambda = \inf_{\rho_2 \in D_{W_2}} \abs{\frac{\langle W_1 \rangle_{\rho_2}}{\langle W_2 \rangle_{\rho_2}}}.
\end{equation}
Так как $W_1$ оптимальнее чем $W_2$, то выполняется $\tilde \lambda \geq 1$, что аналогично следующему неравенству
\begin{equation}
    1 \geq \sup_{\rho_1 \in D_{W_1}}
    \abs{\frac{\langle W_2 \rangle_{\rho_1}}{\langle W_1 \rangle_{\rho_1}}} \geq \lambda \geq 1.
\end{equation}
В последнем неравенстве предполагалось, что $W_2$ оптимальнее, чем $W_1$. Теперь так как $\lambda = 1$ мы получаем равенство $W_1 = W_2$, согласно  четвертому пункту леммы. $\square$.

Далее представим одно из основных свойств, которое поможет нам в дальнейшем. Оно, по сути, говорит нам, что свидетель сцепленности оптимальнее, чем другой, если они отличаются положительным оператором. То есть, если есть один свидетель и мы хотим найти другой, более оптимальный, мы должны вычесть положительный оператор из него.

\begin{lemma}
    $W_2$ оптимальнее, чем $W_1$ тогда и только тогда, когда существует положительно определенный оператор $P \geq 0$ и $1 > \epsilon \geq 0$ такие, что $W_1 = (1 - \epsilon)W_2 + \epsilon P$
\end{lemma}

\underline{Доказательство} Докажем сначала необходимое условие. $\forall \rho \in D_{W_1}$ выполняется $0 > \langle W_1 \rangle_{\rho} = (1-\epsilon) \langle W_2 \rangle_{\rho} + \epsilon \langle W_2 \rangle_{P}$, что означает, что $\rho \in D_{W_2}$. Получаем результат, что $W_2$ оптимальнее, чем $W_1$.

Докажем достаточное условие. Определим аналогичное выражение для $\lambda$ из формулы \ref{lambda-lemma}. Согласное 4-ому пункту леммы 1 $\lambda \geq 1$. В этом случае, если $\lambda = 1$, то $W_1 = W_2$, то есть $\epsilon = 0$. Теперь если $\lambda >1$ определим $P = (\lambda - 1)^{-1}(\lambda W_1 - W_2)$ и $\epsilon = 1-\frac{1}{\lambda} > 0$. При таком выборе $P$ и $\epsilon$ получается равенство аналогичное в лемме $W_1 = (1 - \epsilon)W_2 + \epsilon P$. Остается доказать, что $P \geq 0$, этот факт следует непосредственно из леммы 1 пунктов (1)-(3) $\square$.

\begin{theorem}
    $W$ является оптимальным тогда и только тогда, когда для всех $P \geq 0$ и $\epsilon > 0$ оператор $W^{'} = (1 + \epsilon)W - \epsilon P$ не является свидетелем сцепленности.
\end{theorem}

Существует класс свидетелей сцепленности \textbf{разложимые свидетели сцепленности}. Их можно представить в следующем виде
\begin{equation}\label{decomposal-ew-def}
    W = P + Q^{T_A},
\end{equation}
где $P$ и $Q$ положительно определенные операторы, а $T_A$ транспонирование относительно выделенной партиции бипартиции $A | B$.
Разложимые свидетели сцепленности не могут детектировать сцепленные состояния, но могут обнаружить так называемые \textbf{PPT состояния}.

\section{PPT состояния}
\begin{definition}
    Состояние называется PPT(от англ.  positive partial transpose), когда матрица плотности данного состояния, частично транспонированная относительно партиции $B$ заданной бипартиции $A | B$, является положительно определенной.
    $\rho^{T_B} \geq 0$, где $\rho^{T_B} = (\mathbb{1} \otimes T_B) \rho$.
\end{definition}
Заметим, что любое сепарабельное состояние является PPT состоянием.
Причем этот критерий не работает в обратную сторону, т.е. не любое PPT-состояние является сепарабельным. Можно сказать, что  множество PPT-состояний более обширно, чем множество сепарабельных, но при этом полностью его включает, что схематично показано на рис. \ref{ris:sep-and-ppt-states}.

\begin{figure}[h]
\center{
    \begin{overpic}[width=0.6\textwidth]{sep-and-ppt-states.png}
        \put (40, 55) {PPT-states}
        \put (25, 30) {sep-states}
    \end{overpic}
}
\caption{
Пример как соотносятся PPT-состояния(на рисунке PPT-state) и сепарабельные состояния(на рисунке sep-states). Как можно заметить в силу критерия Переса-Городетского, множество PPT состояний  более обширное и полностью включает множество сепарабельных состояний.
}
\label{ris:sep-and-ppt-states}
\end{figure}

Обозначим PPT-состояние как $\rho_{A | BC}^{ppt}$. Если  заданное состояние можно записано как смесь PPT-состояний, то его называют \textbf{PPT-смешанным}. В случае трех частиц, это условие можно представить следующим образом

\begin{equation}\label{three-ppt-mix-def}
    \rho^{pmix} = p_1 \rho_{A | BC}^{ppt} + 
    p_2 \rho_{B | AC}^{ppt} + 
    p_3 \rho_{C | AB}^{ppt}.
\end{equation}


Очевидно, что любое бисепарабельное состояние является PPT-смешанным. Действительно, если заданное состояние $\rho$ бисепарабельно, то его можно представить в виде \ref{bisep-state-def}, каждое слагаемое в этом разложении является PPT-состоянием. Поэтому в этом случае разложение для бисепарабельного \ref{bisep-state-def} соответствует аналогичному разложению для PPT-смешанного. Таким образом, доказательство, что заданное состояние не представляет собой PPT-смесь, подразумевает, что оно истинно сцепленное. Этот факт лежит в основе алгоритма определения истинной сцепленнсти в данной работе.


Главное преимущество PPT-смешанных состояний перед бисепарабельными заключается в том, что их можно полностью характеризовать  с помощью метода линейного полуопределенного программирования или сокращенно \textbf{SDP}(от англ. semidefinite programming, в дальнейшем будем использовать сокращенное название метода).